% Created 2025-12-10 Wed 09:51
% Intended LaTeX compiler: pdflatex
\documentclass[11pt]{article}
\usepackage[utf8]{inputenc}
\usepackage[T1]{fontenc}
\usepackage{graphicx}
\usepackage{longtable}
\usepackage{wrapfig}
\usepackage{rotating}
\usepackage[normalem]{ulem}
\usepackage{amsmath}
\usepackage{amssymb}
\usepackage{capt-of}
\usepackage{hyperref}
\author{Josune Singaña}
\date{10 de diciembre de 2025}
\title{Informe — Sistema de Recuperación de Información}
\hypersetup{
 pdfauthor={Josune Singaña},
 pdftitle={Informe — Sistema de Recuperación de Información},
 pdfkeywords={},
 pdfsubject={},
 pdfcreator={Emacs 29.3 (Org mode 9.6.15)}, 
 pdflang={English}}
\begin{document}

\maketitle
\tableofcontents


\section{Resumen}
\label{sec:org10ce954}
  Este proyecto implementa un sistema de Recuperación de Información.
El sistema permite indexar un corpus, ejecutar consultas de texto libre y evaluar los resultados mediante métricas estándar.

El corpus utilizado es \textbf{Amazon Fine Food Reviews}, empleando únicamente la columna \texttt{Text} (contenido de la reseña)

\section{Corpus}
\label{sec:org0f92eba}
El corpus contiene miles de reseñas escritas por usuarios. Solo se utiliza el texto principal de cada reseña.
El usuario puede limpiar manualmente el CSV antes de usarlo.

El módulo \texttt{index.py} permite cargar el corpus desde CSV y obtener una lista de documentos.
\section{Preprocesamiento}
\label{sec:org13ff7b4}
Se implementa un \textbf{preprocesamiento fuerte} en \texttt{index.py} mediante la función:

\begin{itemize}
\item Conversión a minúsculas
\item Eliminación de HTML
\item Eliminación de caracteres especiales
\item Normalización de espacios
\item Tokenización por espacios
\item Stopwords básicas
\end{itemize}

El objetivo es obtener tokens representativos y homogéneos.


\section{Diseño}
\label{sec:orgb1208aa}
\section{Construcción del índice}
\label{sec:org9f6875e}
El archivo \texttt{index.py} incluye:
\begin{itemize}
\item Tokenizador
\item Diccionario invertido: término → \{doc: frecuencia\}
\item Cálculo de longitudes documentales
\end{itemize}

Este índice sirve como base para los modelos Jaccard, TF-IDF y BM25.

\section{Modelos de Recuperación}
\label{sec:org8448400}
\subsection{Jaccard (binario)}
\label{sec:org7d1f8f8}
Mide intersección sobre unión entre tokens de consulta y documento.

\subsection{TF-IDF (vectorial)}
\label{sec:org9870a60}
Se utiliza \texttt{TfidfVectorizer} de scikit-learn por eficiencia y robustez.
Las similitudes se calculan con coseno.

\subsection{BM25}
\label{sec:org3ad0be9}
Implementación propia:
\begin{itemize}
\item Parámetros \texttt{k1=1.5} y \texttt{b=0.75}
\item Cálculo de IDF con suavizado
\item Normalización por longitud del documento
\end{itemize}

\section{Interfaces}
\label{sec:orgf8c8dfc}
\subsection{CLI (archivo cli.py)}
\label{sec:orgae8554c}
Permite ejecutar:
\begin{verbatim}
python cli.py --corpus amazon.csv --model bm25 --query "great taste"
\end{verbatim}

\subsection{Interfaz simple (archivo interface\textsubscript{simple.py})}
\label{sec:org99c0b04}
Muestra un menú interactivo que facilita la ejecución de consultas.


\section{Evaluación}
\label{sec:org8ebfd6d}
El módulo \texttt{evaluation.py} permite calcular:
\begin{itemize}
\item Precision@k
\item Recall@k
\item Average Precision (AP)
\item Mean Average Precision (MAP)
\end{itemize}

El archivo \texttt{qrels\_utils.py} permite:
\begin{itemize}
\item Cargar qrels reales
\item Generar qrels de ejemplo
\end{itemize}


\section{Conclusión}
\label{sec:orgb00bd4c}
Este sistema cumple con todas las especificaciones solicitadas:
\begin{itemize}
\item Indexado
\item Preprocesamiento
\item Modelos Jaccard, TF-IDF y BM25
\item Interfaz CLI y menú simple
\item Evaluación con qrels
\end{itemize}
\end{document}
